\documentclass[12pt]{article}

\usepackage[a4paper]{geometry}
\usepackage{graphicx} % Required for inserting images
\usepackage{polski} % do znaków polskich
\usepackage[main=polish,english]{babel}
\usepackage{titlesec} % do edycji formatowania sekcji i podsekcji (\titleformat)
\usepackage{blindtext}
\usepackage[T1]{fontenc}
\usepackage{times} % czcionka podstawowa Times New Roman
\usepackage{indentfirst} % wciecia akapitow
\usepackage{amsmath} 
\counterwithin{equation}{section} %numeracja wzorkow 
\usepackage{float} % to jest potrzebne do [H] przy figure
% [H] obok begin{figure} powoduje ze plik jest na pdfie w tym miejscu jak w 
% kodzie, inaczej ustawia sie on jakos na gorze strony albo inaczej
% niz chce
\usepackage{xcolor} % do zmieniania koloru poszczegolnych blokow tekstu
\usepackage{ragged2e} % do wyrownywania tekstu
%\usepackage[none]{hyphenat} % wylaczenie dzielenia wyrazow na koncu wiersza
\usepackage{caption} % pozwala uzywac \\ w opisach grafik
\usepackage{tabularx} % potrzebne do tabelki z danymi
%\usepackage{fixltx2e} % subscripts bez math mode'a, nie jest potrzebne od 2015 xd
\usepackage[hidelinks]{hyperref} % pozwala na interaktywny spis tresci
\usepackage{gensymb} % m.in. dodaje znak stopni
\usepackage{wrapfig} % do umieszczania obrazow w tekscie
\usepackage{subfig}

% ustawienie formatowania sekcji i podsekcji
\titleformat*{\section}{\normalsize\bfseries}
\titleformat*{\subsection}{\normalsize\bfseries}
\titleformat*{\subsubsection}{\normalsize\bfseries}
\titleformat*{\paragraph}{\large\bfseries}
\titleformat*{\subparagraph}{\large\bfseries}
\geometry{lmargin=3.5cm, rmargin=2.5cm, tmargin=2.5cm, bmargin=2.5cm} % marginesy
\setlength\parindent{1cm} % wciecie 
\renewcommand\thesubsubsection{\alph{subsubsection})} % ustawienie a,b,c...
% jako numeracji podpodsekcji

\usepackage{enumitem}

\title{Odpowiedzi na pytania do wykładu}
\author{Cezary Karolak}
\date{2023Z}

\begin{document}

\maketitle
\tableofcontents
\newpage

\section{Wykład 1}

\begin{enumerate}
    \item{Jakie podstawowe zbiory urządzeń wchodzą w skład systemu elektroenergetycznego?}
    
    \item{W jaki sposób można dokonać podziału jednostek wytwórczych w elektrowniach?}
    
        \textit{Jednostki wytwórcze dzielą się na tradycyjne (węglowe, jądrowe, gazowe, na paliwa płynne) i na wykorzystujące odnawialne źródła energii (wodne, wiatrowe, słoneczne, biogaz i biomasę).}
    
    \item Wymienić parametry charakteryzujące system elektroenergetyczny.

    \item Podać parametry charakteryzujące strukturę mocy systemu elektroenergetycznego.
    \item Podać jakie elementy wchodzą w skład stacji elektroenergetycznej?
    \item Wymienić rodzaje sieci i poziomy napięć występujące obecnie w polskich sieciach elektroenergetycznych.
    \item Wymienić parametry jakościowe energii elektrycznej oraz podać, jaki jest najważniejszy parametr energii elektrycznej, decydujący o pracy systemu elektroenergetycznego?
    \item W którym kraju w Europie jest największy udział OZE w produkcji energii elektrycznej?

        \textit{Największy udział OZE w produkcji energii elektrycznej ma Norwegia - 104,6\%.}
        
    \item Ile państw w Europie wchodzi w skład (jest członkiem) ENTSO-E?

        \textit{W skład ENTSO-E wchodzi 35 państw członkowskich z 39 operatorów systemów przesyłowych. Państwami mającymi status obserwatora są Turcja i Ukraina.}
        
    \item Ile wynosiła moc zainstalowana elektrowni w KSE pod koniec 2022 roku?
    \item Wymienić największe polskie elektrownie na węgiel brunatny, na węgiel kamienny, elektrownie wodne oraz elektrownie gazowe.
    \item Podać ile wynosiła na koniec 2021 roku sumaryczna długość wszystkich linii napowietrznych 400 kV w Polsce.
\end{enumerate}

\section{Wykład 2}

tutaj dodałem jakiś tekst żeby sprawdzić czy git działa ...

\end{document}
