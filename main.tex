\documentclass[12pt]{article}

\usepackage[a4paper]{geometry}
\usepackage{graphicx} % Required for inserting images
\usepackage{polski} % do znaków polskich
\usepackage[main=polish,english]{babel}
\usepackage{titlesec} % do edycji formatowania sekcji i podsekcji (\titleformat)
\usepackage{blindtext}
\usepackage[T1]{fontenc}
\usepackage{times} % czcionka podstawowa Times New Roman
\usepackage{indentfirst} % wciecia akapitow
\usepackage{amsmath} 
\counterwithin{equation}{section} %numeracja wzorkow 
\usepackage{float} % to jest potrzebne do [H] przy figure
% [H] obok begin{figure} powoduje ze plik jest na pdfie w tym miejscu jak w 
% kodzie, inaczej ustawia sie on jakos na gorze strony albo inaczej
% niz chce
\usepackage{xcolor} % do zmieniania koloru poszczegolnych blokow tekstu
%\usepackage{ragged2e} % do wyrownywania tekstu
\usepackage[none]{hyphenat} % wylaczenie dzielenia wyrazow na koncu wiersza
\usepackage{caption} % pozwala uzywac \\ w opisach grafik
\usepackage{tabularx} % potrzebne do tabelki z danymi
\usepackage[hidelinks]{hyperref} % pozwala na interaktywny spis tresci
\usepackage{gensymb} % m.in. dodaje znak stopni
\usepackage{wrapfig} % do umieszczania obrazow w tekscie
\usepackage{subfig}

% ustawienie formatowania sekcji i podsekcji
\titleformat*{\section}{\normalsize\bfseries}
\titleformat*{\subsection}{\normalsize\bfseries}
\titleformat*{\subsubsection}{\normalsize\bfseries}
\titleformat*{\paragraph}{\large\bfseries}
\titleformat*{\subparagraph}{\large\bfseries}
\geometry{lmargin=3.5cm, rmargin=2.5cm, tmargin=2.5cm, bmargin=2.5cm} % marginesy
\setlength\parindent{1cm} % wciecie 
\renewcommand\thesubsubsection{\alph{subsubsection})} % ustawienie a,b,c...
% jako numeracji podpodsekcji

\usepackage{enumitem}
\setlist[enumerate, 1]{label=\textbf{\arabic{*})}} % ustawienie pogrubionej numeracji

\newcommand{\pytanie}[1]{\item \textbf{#1}}

\title{Odpowiedzi na pytania do wykładu}
\author{Cezary Karolak}
\date{2023Z}

\begin{document}

\maketitle
\tableofcontents
\newpage

\section{Wykład 1 -- System elektroenergetyczny}
\begin{enumerate}
    \pytanie{Jakie podstawowe zbiory urządzeń wchodzą w skład systemu elektroenergetycznego?}
    
    \pytanie{W jaki sposób można dokonać podziału jednostek wytwórczych w elektrowniach?}
    
        Jednostki wytwórcze dzielą się na tradycyjne (węglowe, jądrowe, gazowe na paliwa płynne) i na wykorzystujące odnawialne źródła energii (wodne, wiatrowe, słoneczne, biogaz i biomasę).

    \pytanie{Wymienić parametry charakteryzujące system elektroenergetyczny.}
    
        Parametry charakteryzujące SEE:\\
        - moc zainstalowana w jednostkach wytwórczych,\\
        - moc największej jednostki wytwórczej,\\
        - moc największej elektrowni,\\
        - najwyższe napięcie znamionowe (nominalne) sieci przesyłowej,\\
        - zapotrzebowanie szczytowe na moc,\\
        - roczna produkcja energii elektrycznej,\\
        - struktura mocy,\\
        - struktura sieci.

    \pytanie{Podać parametry charakteryzujące strukturę mocy systemu elektroenergetycznego.}
    
        Strukturę mocy charakteryzuje sposób pokrywania obciążeń systemu, zawiera więc dane o jednostkach wytwórczych.

    \pytanie{Podać jakie elementy wchodzą w skład stacji elektroenergetycznej?}
    
        W stacjach elektroenergetycznych występują następujące elementy:\\
        - transformatory,\\
        - szyny zbiorcze,\\
        - łączniki,\\
        - dławiki,\\
        - baterie kondensatorów,\\
        - inne urządzenia.

    \pytanie{Wymienić rodzaje sieci i poziomy napięć występujące obecnie w polskich sieciach elektroenergetycznych.}

        Sieci elektroenergetyczne dzieli się na sieci energetyki zawodowej i sieci energetyki przemysłowej.\\ Sieci elektroenergetyczne dzieli się na sieci przesyłowe i dystrybucyjne. Sieci przesyłowe to stacje najwyższych napięć (NN), czyli linie o napięciu powyżej 110kV (220kV i 400kV). Sieci dystrybucyjne to stacje niskich (nN), średnich (SN) i wysokich (WN) napięć. W Polsce są to:\\
        - nN: 0,4kV i 0,66kV\\
        - SN: 6kV, 10kV, 15kV, 20kV i 30kV\\
        - WN: 110kV
    
    \pytanie{Wymienić parametry jakościowe energii elektrycznej oraz podać, jaki jest najważniejszy parametr energii elektrycznej, decydujący o pracy systemu elektroenergetycznego?}

        Podstawowe parametry jakościowe energii elektrycznej to:\\
        - częstotliwość,\\
        - napięcie (wartość, poziom),\\
        - symetria fazowa napięć,\\
        - zawartość harmonicznych w krzywej napięcia,\\
        - ciągłość dostawy energii.\\
        Najważniejszy parametr jakościowy to częstotliwość. Zależy ona od zapasu mocy czynnej w jednostkach wytwórczych oraz od działania układów automatycznej regulacji częstotliwości w SEE.

    \pytanie{W którym kraju w Europie jest największy udział OZE w produkcji energii elektrycznej?}

        Największy udział OZE w produkcji energii elektrycznej ma Norwegia - 104,6\%.
        
    \pytanie{Ile państw w Europie wchodzi w skład (jest członkiem) ENTSO-E?}

        W skład ENTSO-E wchodzi 35 państw członkowskich z 39 operatorów systemów przesyłowych. Państwami mającymi status obserwatora są Turcja i Ukraina.
        
    \pytanie{Ile wynosiła moc zainstalowana elektrowni w KSE pod koniec 2022 roku?}
    
        Moc zainstalowana elektrowni (łączna moc znamionowa wszystkich jednostek wytwórczych) w KSE (Krajowy System Elektroenergetyczny) pod koniec 2022 roku wynosiła 60446 MW.

    \pytanie{Wymienić największe polskie elektrownie na węgiel brunatny, na węgiel kamienny, elektrownie wodne oraz elektrownie gazowe.}
    
        Największe polskie elektrownie:\\
        - węgiel brunatny: Bełchatów (5102 MW),\\
        - węgiel kamienny: Kozienice 1+2 (4020 MW),\\
        - wodna pompowa: Żarnowiec (780 MW),\\
        - gazowa: Płock (630 MW).

    \pytanie{Podać ile wynosiła na koniec 2021 roku sumaryczna długość wszystkich linii napowietrznych 400 kV w Polsce.}
    
        Na koniec 2021 roku sumaryczna długość wszystkich linii napowietrznych 400kV w Polsce wynosiła 8227 km.

\end{enumerate}

\newpage

\section{Wykład 2 -- Jednostki wytwórcze energii elektrycznej}
\begin{enumerate}
    \pytanie{Omówić cykl przemian energii w elektrowni cieplnej.}

        W wyniku spalania paliwa organiczego lub rozszczepienia paliwa jądrowego otrzymuje się energię cieplną,którą przekazujemy do czynnika roboczego (zwykle czynnikiem jest para wodna). Czynnik roboczy wykonuje pracę w silniku cieplnym napędzając tym samym generator prądu. W większości przypadków rolę silnika cieplnego pełni turbina parowa.

    \pytanie{Podać zasadnicze różnice między układem jednoobiegowym i dwuobiegowym w elektrowni jądrowej.}
    
        W układzie jednobiegowym chłodziwo reaktora (to co odbiera ciepło reaktora) jest również czynnikiem roboczym w turbinie (napędza generator).\\
        W układzie dwubiegowym chłodziwo reaktora przekazuje ciepło czynnikowi roboczemu, a elementem sprzęgającym oba biegi jest wymiennik ciepła (wytwornica pary).

    \pytanie{Podać zalety stosowania elektrowni wodnych w stosunku do elektrowni cieplnych.}

        Zalety stosowania elektrowni wodnych w stosunku do elektrowni cieplnych:\\
        - wykorzystują odnawialne zasoby energetyczne przyrody,\\
        - posiadają wysoką sprawność, często przekraczającą 90\%,\\
        - wywierają stosunkowo najmniej destruktywny wpływ na środowisko,\\
        - charakteryzują się szybkim rozruchem i możliwością niemal natychmiastowego pełnego obciążenia.
    
    \pytanie{Przedstawić podział elektrowni wodnych ze względu na ich konstrukcję.}
    
        Elektrownie wodne ze względu na konstrukcję dzieli się na:\\
        - \underline{przepływowe}: wykorzystują ciągły przepływ wody, nie posiadają zbiornika do jej magazynowania\\
        - \underline{zbiornikowe}: wyposażone w duże zbiorniki umożliwiające gromadzenie dużej ilości wody\\
        - \underline{szczytowo - pompowe}: pełnią rolę magazynów energii. W szczycie obciążenia generują energię elektryczną w wyniku przepływu wody ze zbiornika górnego do dolnego. W okresie zmniejszonego zapotrzebowania na moc przepompowują wodę ze zbiornika dolnego do górnego.

    \pytanie{Sklasyfikować jednostki generacji rozproszonej pod względem mocy zainstalowanej.}

        Podział jednostek generacji rozproszonej pod względem mocy zainstalowanej:\\
        - mikrogeneracja rozproszona - od 1 W do 5 kW\\
        - mała generacja rozproszona - od 1 kW do 5 MW\\
        - średnia generacja rozproszona - od 5 MW do 50 MW\\
        - duża generacja rozproszona - od 50 MW do 150 MW
        
    \pytanie{Podać główne zalety i wady elektrowni wiatrowych.}

        Zalety elektrowni wiatrowych:\\
        - wykorzystanie niewyczerpalnych zasobów energii\\
        - brak emisji szkodliwych substancji do środowiska\\
        Wady elektrowin wiatrowych:\\
        - emisja hałasu podczas pracy\\
        - wywołanie zakłóceń elektromagnetycznych\\
        - duża zajętość terenu\\
        - szpecenie krajobrazu\\
        - zagrożenie dla ptaków\\
        - praca uzależniona od prędkości wiatru
        
    \pytanie{Wyjaśnić różnicę pomiędzy systemami fotowoltaicznymi off-grid i ongrid.}

        Systemy fotowoltaiczne off-grid to systemy autonomiczne, współpracujące z magazynem energii. Systemy fotowoltaiczne ongrid są przyłączone do sieci dystrybucyjnej za pomocą falownika.

    \pytanie{Wyjaśnić zasadę działania i cechy charakterystyczne spalinowego silnika tłokowego.}

        Silniki tłokowe mogą być zasilane olejem napędowym, etyliną, gazem ziemnym, gazem LPG oraz biogazem. Zasada działania - przez spalanie paliwa w cylindrach silnika powstaje energia cieplna, która zostaje zamieniona na energię mechaniczną. Spalana w cylindrach mieszanka paliwowa napędza tłoki silnika, które przekazują energię na wał korbowy.\\ Silniki tłokowe charakteryzują się niewielkimi mocami, modularnością, niskimi potrzebnymi nakładami kapitałowymi oraz niskimi kosztami eksploatacji. Mają również krótki czas rozruchu oraz dużą niezawodność. Jednostki tego typu są często stosowane jako awaryjne źródła mocy lub podstawowe źródła niewielkich systemów elektroenergetycznych (mikrosystemów).
        
    \pytanie{Narysować i omówić schemat ideowy przyłączenia mikroturbiny do sieci elektroenergetycznej.}

    Z uwagi na wysoką prękość obrotową generatora napędzanego mikroturbiną, wytwarzane napięcie ma wysoką częstotliwość. Aby móc przyłączyć tego typu źródła do sieci elektroenergetycznej należy wykorzystać przekształtnik energoelektroniczny.

    \begin{figure}[H]
        \centering
        \includegraphics[width=0.9\linewidth]{images/wyklad2_1.png}
        \caption*{Przyłączenie mikroturbiny do sieci}
    \end{figure}

\end{enumerate}

\newpage

\section{Wykład 3 -- Odbiorcy, odbiory i odbiorniki energii elektrycznej}
    \begin{enumerate}
        \pytanie{Podaj przykładowe podziały odbiorników energii elektrycznej na poszczególne kategorie.}
        \pytanie{Czym charakteryzują się domowi odbiorcy energii elektrycznej?}
        \pytanie{Przedstaw typowe kształty w profilach obciążenia domowych odbiorców energii elektrycznej.}
        \pytanie{Czym charakteryzują się przemysłowi odbiorcy energii elektrycznej?}
        \pytanie{Przedstaw tryby ładowania pojazdu elektrycznego.}
        \pytanie{Przedstaw czynniki wpływające na obciążenie stacji ładowania pojazdów elektrycznych.}
        \pytanie{Przedstaw fazy ładowania akumulatora litowo-jonowego pojazdu elektrycznego.}
        \pytanie{Porównaj tramwaj i trolejbus pod względem energochłonności.}
        \pytanie{Od czego zależy pobór energii elektrycznej przez tramwaj?}
        \pytanie{Przedstaw zagadnienie energii hamowania dla pojazdów metra.}
        \pytanie{Od czego zależy zapotrzebowanie na energię elektryczną przez kolej elektryczną?}
    \end{enumerate}


\section{Wykład 7 -- Moc i energia w systemie elektroenergetycznym}
	\begin{enumerate}
	    \pytanie{Przedstaw trójkąt mocy i wynikające z niego zależności pomiędzy występującymi w nim mocami.}
	
	        \begin{figure}[H]
	            \centering
	            \includegraphics[width=0.3\linewidth]{images/wyklad7_1.png}
	            \caption*{Trójkąt mocy}
	        \end{figure}
	        \begin{equation*}
	            P = S \cdot \cos\varphi
	        \end{equation*}
	        \begin{equation*}
	            Q = S \cdot \sin\varphi
	        \end{equation*}
	        \begin{equation*}
	            S = P + jQ
	        \end{equation*}
	
	    \pytanie{Wyjaśnij, z czym silnie związane są zmiany mocy czynnej i biernej w systemie 
	    elektroenergetycznym.}
	
	        Wpływając na wartość mocy czynnej jednocześnie wpływa się mocno na kąt mocy ($\delta$) i odwrotnie.
	        Wpływając na wartość mocy biernej jednocześnie wpływa się na wartość napięcia.
	        Wynika to z poniższych wzorów:
	        \begin{equation*}
	            P = U I \cos\varphi = \frac{E U}{X} \sin\delta
	        \end{equation*}
	        \begin{equation*}
	            Q = U I \sin\varphi = \frac{E U}{X} \cos\delta - \frac{U^2}{X}
	        \end{equation*}
	
	    \pytanie{Przedstaw główne odbiorniki mocy biernej.}
	
	        
	
	    \pytanie{Przedstaw model obwodowy linii elektroenergetycznej wysokiego napięcia.}
	
	    \pytanie{Do czego służy energetyczny równoważnik mocy biernej?}
	
	    \pytanie{Przedstaw negatywne skutki przesyłania mocy biernej przez sieć 
	    elektroenergetyczną.}
	
	    \pytanie{Wyjaśnij, za pomocą odpowiedniego wzoru, dlaczego obecność mocy biernej 
	    zmniejsza zdolność przepustową linii elektroenergetycznych.}
	
	    \pytanie{Przedstaw naturalne i sztuczne metody kompensacji mocy biernej.}
	
	    \pytanie{Podaj zalety i wady stosowania kondensatorów do kompensacji mocy biernej.}
	
	    \pytanie{Przedstaw możliwe rodzaje kompensacji mocy biernej w zależności od miejsca 
	    instalacji baterii kondensatorów.}
	
	    \pytanie{Który rodzaj kompensacji mocy biernej jest najbardziej skuteczny, a który 
	    najmniej? Odpowiedź uzasadnij.}
	
	    \pytanie{Jaki jest wzór na moc baterii do centralnej kompensacji mocy biernej?}
	
	    \pytanie{Przedstaw metody zmniejszania strat mocy w sieci elektroenergetycznej.}
	
	\end{enumerate}



\end{document}
