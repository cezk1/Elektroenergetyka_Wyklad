\usepackage[a4paper]{geometry}
\usepackage{graphicx} % Required for inserting images
\usepackage{polski} % do znaków polskich
\usepackage[main=polish,english]{babel}
\usepackage{titlesec} % do edycji formatowania sekcji i podsekcji (\titleformat)
\usepackage{blindtext}
\usepackage[T1]{fontenc}
\usepackage{times} % czcionka podstawowa Times New Roman
\usepackage{indentfirst} % wciecia akapitow
\usepackage{amsmath} 
\counterwithin{equation}{section} %numeracja wzorkow 
\usepackage{float} % to jest potrzebne do [H] przy figure
% [H] obok begin{figure} powoduje ze plik jest na pdfie w tym miejscu jak w 
% kodzie, inaczej ustawia sie on jakos na gorze strony albo inaczej
% niz chce
\usepackage{xcolor} % do zmieniania koloru poszczegolnych blokow tekstu
\usepackage{ragged2e} % do wyrownywania tekstu
%\usepackage[none]{hyphenat} % wylaczenie dzielenia wyrazow na koncu wiersza
\usepackage{caption} % pozwala uzywac \\ w opisach grafik
\usepackage{tabularx} % potrzebne do tabelki z danymi
%\usepackage{fixltx2e} % subscripts bez math mode'a, nie jest potrzebne od 2015 xd
\usepackage[hidelinks]{hyperref} % pozwala na interaktywny spis tresci
\usepackage{gensymb} % m.in. dodaje znak stopni
\usepackage{wrapfig} % do umieszczania obrazow w tekscie
\usepackage{subfig}

% ustawienie formatowania sekcji i podsekcji
\titleformat*{\section}{\normalsize\bfseries}
\titleformat*{\subsection}{\normalsize\bfseries}
\titleformat*{\subsubsection}{\normalsize\bfseries}
\titleformat*{\paragraph}{\large\bfseries}
\titleformat*{\subparagraph}{\large\bfseries}
\geometry{lmargin=3.5cm, rmargin=2.5cm, tmargin=2.5cm, bmargin=2.5cm} % marginesy
\setlength\parindent{1cm} % wciecie 
\renewcommand\thesubsubsection{\alph{subsubsection})} % ustawienie a,b,c...
% jako numeracji podpodsekcji
